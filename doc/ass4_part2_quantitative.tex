% 
% Information Retrieval
% -------------------------------
% University of California Irvine
% 
% Author:
% - María Carrasco Rodríguez
% - Fabian Lindenberg
% - Lea Voget

\documentclass[a4paper,11pt,oneside]{book}
\usepackage{wrapfig} 
\usepackage{helvet}
\usepackage{phdthesis}
\usepackage{kostspielig}
\usepackage{amsmath}
\usepackage{multirow}
\usepackage{colortbl}
\usepackage{appendix}

\usepackage{color}
\usepackage{amsmath}
\usepackage{epsfig}


%\pagestyle{fancy}
%\fancyhead{}
%\fancyhead[L]{\textsf{\textbf{CS 221 Information Retrieval}\\ Assignment 2}}
%\fancyhead[R]{\textsf{Maria Carrasco (16874129)\\ Fabian Lindenberg (74076658)}}
%\fancyfoot[R]{\textsf{\thepage}}
%\renewcommand{\footrulewidth}{0.4pt}
%\cfoot[]{}


\newcommand{\todo}[1]{\textcolor{red}{\textbf{TODO: #1}}}

\hypersetup{colorlinks, 
           linkbordercolor= 1 0.8 0.8,
           citecolor=black,
           filecolor=black,
           linkcolor=black,
           urlcolor=black,
           bookmarksopen=true,
           pdftex}
\title{Information Retrieval }
\subtitle{Assignment 4}
\location{University of California Irvine}
\author{ María Carrasco Rodríguez (16874129) \\
		Fabian Lindenberg (74076658) \\
		Lea Voget (45869178)}



\begin{document}

\kostspieligmaketitle

%\tableofcontents
%\pagebreak
% \startapendix

\setcounter{chapter}{1}
\chapter{Quantify the data}

\begin{enumerate}
	\item \emph{Comment:} Unfortunately, our installed Antivirus software\footnote{Avira Antivir Personal 10} reported two of the files contained in the archive as being infected by viruses and put them under quarantine. Consequently, the following results may vary slightly from the true numbers.
		\begin{enumerate}
			\renewcommand{\labelenumii}{\alph{enumii})}
			\item \label{people} Number of people targeted in the Enron data set: $150$
			\item \label{individual} Number of individual data files: $517,428$
			\item \label{sent}Number of sent data files: $128,462$
			\item We consider Inboxes all folders that do not contain sent e-mails. According to this definition, the number of data files in inboxes is, thus, the total number of files minus the number of sent data files: $517,428 - 128,462 = 388,966$
			\item \label{top} The ten persons with the largest number of files are in descending order ``kaminski-v'', ``dasovich-j'', ``kean-s'', ``mann-k'', ``jones-t'', ``shackleton-s'', ``taylor-m'', ``farmer-d'', ``germany-c'', and ``beck-s''.
		\end{enumerate}
	\item To quantify the data, we wrote a Java program that iterated over the files and subfolders. For subtask \ref{people}), the number of folders in the first hierarchy level were counted. To count all individual data files in subtask \ref{individual}), the program scanned \emph{recursively} through all subfolders. For subtask \ref{sent}), only files in folders whose names matched the \emph{regular expression} \texttt{(.*[$\,\hat{}$A-Za-z])?sent([$\,\hat{}$A-Za-z].*)?} were counted. To solve subtask \ref{top}), we maintained a sorted mapping between number of files (key) and a list of folder names (value), and returned the first ten values. If multiple folders of equal size were ranked tenth, all these folders would be returned.
\end{enumerate}

\end{document}
